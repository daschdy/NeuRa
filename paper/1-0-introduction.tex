\chapter{Introduction}

As part of the seminar “Neuartige Rechnerarchitekturen”, the paper “Flexible and Efficient QoS Provisioning in AXI4-Based Network-on-Chip Architecture” by Wang and Lu (2022) is analyzed in detail.\cite{wang_flexible_2022} The goal of this analysis is to engage deeply with the presented architecture, identify key concepts and findings, and prepare a seminar presentation that clearly conveys these to fellow students.
The seminar is part of the Bachelor's program in Computer Science at Friedrich-Alexander-Universität Erlangen-Nürnberg and was attended during the summer semester of 2025.

The title of the paper already points to three essential core aspects: \ac{QoS}, \ac{AXI4}, and \ac{NoC}. These terms will be explained in more detail, placed in their technical context, and examined in relation to each other. The objective is to provide a comprehensive overview and analysis of the architecture developed by Wang and Lu.

With the increasing number of processor cores and functional units in modern \ac{SoC} designs, the demands on efficient and reliable communication between these components also grow. For this reason, NoC architectures are becoming increasingly important. At the same time, many applications require low latencies and guaranteed bandwidth, making flexible QoS support essential. The AXI4 standard has become a widely adopted protocol in the industry, highlighting the relevance and timeliness of the approach presented by Wang and Lu.\cite{jake_ke_demystifying_2025}\cite{gomez-rodriguez_survey_2021}\cite{talwar_traffic_2013}

To provide technical context, the key concepts (\acs{NoC}, \acs{QoS}, and ˜\acs{AXI4}) are first compared with the current state of the art and relevant research literature. This is followed by a detailed description of the architecture introduced by Wang and Lu, including discussion of the role of the subnetworks (\ac{VC} and \ac{TDM}) and the \ac{NI}. Building on this, the experimental results are presented and critically evaluated.
