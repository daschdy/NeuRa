\section{Quality of Service (QoS)}

\sh{What is Quality of Service in Networking?}
\ac{QoS} refers to a set of technologies and techniques used in networking to manage traffic and ensure the efficient and predictable performance of critical applications. It allows organizations to prioritize certain types of traffic over others, ensuring that resource-intensive or time-sensitive services maintain high performance even under constrained bandwidth conditions. QoS is particularly important in networks that handle real-time data, such as \ac{IPTV}, online gaming, media streaming, video conferencing, \ac{VoD}, and \ac{VoIP}. By applying QoS policies, organizations can optimize the behavior of multiple applications, gaining visibility and control over network characteristics such as bit rate, jitter (see below at \ref{jitter}), packet loss, and latency.\cite{rhim_what_2024}\cite{hpe_juniper_networking_what_nodate}\cite{paloalto_networks_what_nodate}\cite{fortinet_what_nodate}


\sh{Types of Traffic}
Different types of traffic are affected by various network parameters that influence performance. \textit{Bandwidth} refers to the maximum rate at which data can be transferred across the network, while throughput represents the actual rate achieved. \textit{Latency} is the delay experienced in transmitting data from source to destination. \textit{Jitter}, on the other hand, refers to the variation in packet arrival times, often caused by network congestion. This variation can lead to packets arriving late or out of sequence, affecting the quality of real-time applications such as voice and video.\label{jitter}\cite{paloalto_networks_what_nodate}\cite{fortinet_what_nodate}


\sh{How Does QoS Work?}
As businesses increasingly rely on networks to transmit information between endpoints, data is divided into packets for transmission. These packets, much like letters in envelopes, are routed through the network. Since bandwidth is limited, QoS is responsible for determining which packets receive priority to ensure that critical traffic is delivered reliably and efficiently. QoS achieves this by classifying traffic based on predefined policies and assigning priorities to different classes of traffic. High-priority packets, such as those carrying voice or video, are given preferential treatment over less critical traffic like file downloads or email.

The implementation of QoS involves several key mechanisms. Traffic classification and marking are used to identify and label packets according to their type or importance. Queuing mechanisms then determine the order in which packets are transmitted, using techniques such as priority queuing or weighted fair queuing. Bandwidth management ensures that different traffic types receive appropriate resource allocation, while policing and shaping tools are used to enforce traffic limits or smooth traffic flows. Congestion management techniques help to prevent buffer overflow and packet loss during periods of high network usage. These combined techniques ensure that network resources are allocated efficiently and in accordance with organizational priorities.\cite{paloalto_networks_what_nodate}\cite{hpe_juniper_networking_what_nodate}\cite{rhim_what_2024}


\sh{QoS Models}
QoS can be implemented using several different models. The best-effort model provides no guarantees and treats all traffic equally, making it suitable only for non-critical applications.\cite{bruno_wan_2024} The \ac{IntServ} model offers strict QoS guarantees by reserving network resources for specific traffic flows using signaling protocols such as the Resource Reservation Protocol (RSVP).\cite{networklessons_introduction_nodate} In contrast, the \ac{DiffServ} model is more scalable and widely used in modern enterprise networks. DiffServ classifies and manages traffic into different service levels without requiring end-to-end signaling, allowing more flexible and efficient QoS implementation.\cite{bruno_wan_2024}
