\section{Related Work}

% Keep this chapter shorter
\iffalse
\sh{AXI4-Based Communication Support}
Several studies have attempted to adapt the AXI4 protocol to NoC-based communication. For instance, Kwon et al.\cite{yang_nisar_2007} utilized in-network reorder buffers to meet ordering requirements, while Yang et al.\cite{kwon_-network_2009} designed AXI-compliant network interfaces for transaction reordering. Hybrid and debug-aware network interfaces have also been proposed.

Other works explored FPGA-based AXI-NoC integration or backward compatibility with existing bus protocols. However, these approaches mostly focus on transaction ordering and do not address the crucial problem of message format conversion between AXI4 signals and NoC packets. Furthermore, they lack support for multiple QoS schemes, which are essential for diverse application demands.\footnote{For further related work, see Wang \& Lu [1, p. 1524, Sec. II-A].} 


\sh{QoS Provisioning}
QoS provisioning in NoCs has been widely studied, yet most solutions support only one or two QoS schemes. Sharifi et al.\cite{sharifi_addressing_2012} prioritized packets targeting idle memory banks, Chen et al.\cite{chen_round-trip_2017} introduced DRAM round-trip latency prediction to prioritize critical flows, and Liu et al.\cite{liu_highway_2015} proposed highway-based TDM NoCs for throughput improvements. Goossens et al.\cite{goossens_aethereal_2005} combined guaranteed and best-effort services. 

Nevertheless, these schemes remain limited and do not fully capture the multiple QoS requirements of AXI4-based systems, such as latency-critical CPUs, bandwidth-hungry GPUs, and best-effort I/O devices. 


\sh{Flow Control}
Flow control determines how resources are allocated across the network and can be implemented in routers or at endpoints. Router-based methods include adaptive \ac{VC} reallocation, flit bypassing, and path pre-reservation. Admission control approaches range from $(\sigma,\rho)$-based regulation to fuzzy-control mechanisms and ANN-based admission control. Although effective in optimizing latency and throughput, these methods are not specifically optimized for AXI4-based NoCs with heterogeneous QoS demands.\footnote{For further related work, see Wang \& Lu [1, p. 1524, Sec. II-C].}  


\sh{TDM Routing Algorithm}
Routing in TDM-based NoCs is a key challenge, where both path selection and time-slot allocation must be satisfied. Different strategies include two-step backtracking (Lu and Jantsch \cite{lu_tdm_2008}), multi-iteration routing (Patil et al. \cite{patil_bandwidth-optimized_2018}), greedy and pathfinder algorithms (Kapre et al. \cite{kapre_packet_2006}), offline latency-based scheduling, distributed routing using slot tables, and probe-based runtime allocation. These algorithms balance predictability and adaptability, but none of them target the unique QoS requirements of AXI4 traffic. 


\sh{Router Microarchitecture Design}
Router design strongly influences latency and throughput in NoCs. Proposals include distributed shared-buffer routers emulating output-buffer routers, low-latency wormhole routers, dynamic look-ahead bypass routers, fault-tolerant router designs, and lightweight runahead NoCs.

Although these works improve efficiency, they primarily target generic NoCs and do not integrate AXI4-specific requirements or multi-QoS considerations.lly optimized for AXI4-based NoCs with heterogeneous QoS demands.\footnote{For further related work, see Wang \& Lu [1, p. 1524, Sec. II-E].} 


\sh{Industrial Designs for AXI4-Based QoS}
Industry has developed several QoS-enabled AXI4-based systems. Xilinx’s Zynq-7000 devices provide per-transaction priority mechanisms, Synopsys DesignWare integrates QoS regulators and arbiters, and Intel’s Qsys interconnect uses flow controllers and bandwidth regulators. In the 5G domain, AXI4-compatible NoCs have been applied in baseband processing and low-latency switching devices.

However, these industrial designs are often vendor-specific and do not provide detailed architectural insights or flexible multi-QoS support.\footnote{For further related work, see Wang \& Lu [1, p. 1524, Sec. II-F].}


\sh{Compared With Wang \& Lu's Previous Work}
This article extends the authors’ earlier six-page conference paper. The initial work introduced the novel network interface design, the definition of QoS services, and a system simulator.

The referred article enhances this by proposing a traffic conversion mechanism for balancing \ac{VC} and \ac{TDM} subnetworks, exploring multiple flow control schemes, and introducing a two-level \ac{MMP}-based traffic generator for realistic workloads.

Thus, it provides a more complete discussion, expanded experiments, and improved performance results.\footnote{For further related work, see Wang \& Lu [1, p. 1524, Sec. II-G].} 
\fi
% Now its gettin' interessting

\sh{Recent Publications}
Since the authors provided an overview of related work within their paper (see Wang \& Lu [p. 1524--1526, Sec. II]), therefore this chapter focuses on several new contributions that are closely related to AXI4-based NoCs and QoS-aware architectures.

PATRONoC introduces an open-source, fully AXI4-compliant NoC fabric specifically designed for multi-accelerator \ac{DNN} platforms. It demonstrates up to 34\% improved area efficiency and achieves between two- and eight-fold throughput improvements compared to state-of-the-art designs.\cite{jain_patronoc_2023} 

Another line of research is represented by FlooNoC, which has appeared in two iterations. The 2023 version presents a low-latency, wide-channel AXI4-compatible NoC, achieving 629\,Gbps per link at 1.23\,GHz in 12\,nm \ac{FinFET}\footnote{FinFET is a 3D transistor structure where the conducting channel is formed in a thin vertical “fin” of silicon. Unlike traditional planar MOSFETs, its gate wraps around multiple sides of the fin, providing better electrostatic control, reducing leakage current, and enabling higher switching speeds and energy efficiency at advanced technology nodes (e.g., 12 nm, 7 nm).} technology with only 10\% area overhead.\cite{fischer_floonoc_2023} 

An extended 2024 version further enhances performance by reaching 645\,Gbps per link, 103\,Tbps aggregate bandwidth, and energy efficiency of 0.15\,pJ/B per hop, while still maintaining very low hardware overhead.\cite{fischer_floonoc_2025}

AXI-REALM (2023--2024) provides a real-time extension for AXI4 interconnects, introducing credit-based traffic regulation and observability of per-manager traffic. When implemented in a Linux-capable RISC-V SoC, it reduces worst-case memory latency from over 264 cycles to fewer than 8 cycles, with an area overhead as low as 2.45\%.\cite{benz_axi-realm_2023}\cite{benz_axi-realm_2025}

From the perspective of QoS-aware router design, AQ-BiNoC introduces an anticipative mechanism that improves the latency of high-priority \ac{GS} packets by roughly 14--35\% compared to traditional NoC and \ac{BiNoC}\footnote{A BiNoC is a NoC architecture in which communication channels between routers can dynamically switch direction. Instead of having two fixed unidirectional links per connection, a single physical link can be time-multiplexed to carry data either way, depending on traffic demand.} routers under various traffic scenarios.\cite{tsai_anticipative_2022} 

Finally, reliability aspects of AXI4 have been addressed in 2025 with the proposal of a \ac{TMU} for AXI4 interconnects, which enables real-time detection of protocol violations and timeout faults. The \ac{TMU} supports monitoring of up to 32 outstanding transactions, offering different levels of granularity to balance coverage and cost.\cite{liang_towards_2025}
