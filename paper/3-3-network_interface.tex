\section{Network Interface (NI)}

\sh{Structure and Function}
The NI bridges AXI4-based masters/slaves and the packet-switched NoC. It converts AXI4 signals into packets suitable for the target subnetwork and, conversely, reconstructs AXI4 transactions from received packets. This makes the NoC protocol-agnostic while remaining fully AXI4-compliant.

\sh{Conversion of AXI4 Transactions}
The NI performs message format conversion between AXI4 transactions and NoC packets. 
Two approaches are possible: 
(i) direct mapping of the five AXI4 channels into the NoC, or 
(ii) protocol adaptation, where AXI4 requests and responses are translated into unified packet formats. 
The second approach is adopted in this design, since it decouples the NoC from the AXI4 protocol and enables more efficient packet-based communication. 

Packets with LCS and URS QoS identifiers are encapsulated as VC packets and transmitted in the VC subnetwork, while GRS packets are encapsulated as TDM packets and transmitted in the TDM subnetwork. 
This mapping ensures that each AXI4 transaction is routed through the most appropriate subnetwork according to its QoS requirements.

\sh{QoS Inheritance}
Because AXI4 response signals do not include QoS identifiers, the NI implements a QoS inheritance mechanism: the response packets automatically inherit the QoS information of their corresponding request packets. This ensures consistent QoS handling across request/response transactions.

\sh{Traffic Converter}
A key element of the NI is the \textit{Traffic Converter}, which dynamically balances the load between the two subnetworks. 
\begin{itemize}
    \item When the VC subnetwork is congested, selected LCS packets can be redirected to the TDM subnetwork.
    \item Conversely, if the TDM subnetwork is heavily loaded, GRS packets can be offloaded to the VC subnetwork, without violating bandwidth guarantees.
\end{itemize}
This adaptive mechanism improves overall utilization, reduces latency under congestion, and enhances throughput while maintaining QoS guarantees.
